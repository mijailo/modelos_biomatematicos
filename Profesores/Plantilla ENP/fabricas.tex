\documentclass[letterpaper,11pt]{article}
\usepackage[margin=1.5cm]{geometry}
\usepackage[spanish]{babel}
\usepackage[utf8]{inputenc}
\usepackage[T1]{fontenc}
\usepackage{times}
\usepackage{calligra}
\usepackage{graphicx}
\usepackage{latexsym}
\usepackage{amsmath,amssymb}
\usepackage{subfigure}
\usepackage{tabulary}
\usepackage{url}
\spanishdecimal{.}

\usepackage{float}
\usepackage{colortbl}
\usepackage{latexsym}
\usepackage{xcolor}
\usepackage{fancyhdr}
\pagestyle{fancy}
\usepackage{balance}
\usepackage[colorlinks=true, allcolors=blue]{hyperref}


%\begin{figure}[t!]
%\flushleft  \includegraphics[width=1in]{ENPlogo.jpg}
%\end{figure}

\title{Fábrica de empaques reciclables\\
Aulas virtuales}
\author{Olivia Palma Avendaño \\ \href{mailto:
oliviapalma@gmail.com}{oliviapalma@gmail.com}
}

\date{}

\begin{document}
\maketitle


\begin{center}
 % \textit{Escuela Nacional Preparatoria No. 5 José Vasconcelos\\
%Universidad Nacional Autónoma de México}


	%\vspace*{0.1cm}
	
	\begin{center}\rule{0.9\textwidth}{0.1mm} \end{center}

  
\end{center}

\begin{enumerate}

\item Un cliente solicita a una fábrica que diseñe una caja de cartón sin tapa para colocar galletas, que contenga el mayor volumen posible usando una lámina de cartón reciclado de $20\,cm$ por $30\,cm$.

    \begin{enumerate}
        \item ¿Encuentras algunas variables involucradas en la solución?
        \item ¿Existe una relación o ecuación que se pueda plantear para resolver esta situación?
        \item Plantea la o las ecuaciones o funciones.
        \item Grafica y describe las medidas de la caja que diseñaste.
    \end{enumerate}
    
\item  Vas a construir un armario con cuatro entrepaños en una pared de $2.2\,m$ de altura, para que la distancia entre ellos aumente $13$ $cm$ de arriba hacia abajo. El espesor de cada entrepaño es de $2.5\,cm$.

    \begin{enumerate}
        \item ¿A qué distancia del techo quedará el entrepaño superior?
        \item ¿Cómo lo vas a resolver?
        \item ¿Es necesario que hagas un dibujo o esquema de la situación?
        \item ¿Cuáles son las variables involucradas?
        \item ¿Encuentras alguna relación entre estas variables? Es decir, una ecuación o función. Si es así, escríbela.
        \item ¿Cómo planteas la solución para el diseño de este armario?
        \item  Escribe las funciones involucradas y grafica.
    \end{enumerate}
    
\item Vas a diseñar portarretratos de ancho uniforme de $2.5 \, cm$ con un área de cartón de $175\,cm^2$. Lo más importante es que estos portarretratos enmarquen fotografías de la \textit{mayor} área posible.

    \begin{enumerate}
        \item ¿Encuentras alguna diferencia entre este pedido y los dos anteriores?
        \item Realiza un esquema o dibujo que represente la situación.
        \item ¿Cuáles son las variables involucradas en la situación?
        \item ¿Cuáles son las relaciones entre las variables involucradas?
        \item ¿Cómo lo resolverás?
        \item ¿Cómo lo hiciste?
    \end{enumerate}

\item Vamos a construir un empaque cilíndrico con tapa que contenga $1000 \, mL$ de tal forma que se ocupe la menor cantidad de material posible para su construcción.
    
    \begin{enumerate}
        \item ¿Cuáles son las variables involucradas?
        \item ¿Cuál es la variable independiente y cuál es la dependiente?
        \item ¿Cómo encuentras las medidas óptimas?
        \item Escribe la función de área. Grafica y determina las medidas del empaque cilíndrico.
    \end{enumerate}
        
\end{enumerate}

\end{document}
