\documentclass[letterpaper,11pt]{article}
\usepackage[margin=2cm]{geometry}
\usepackage[spanish]{babel}
\usepackage[utf8]{inputenc}
\usepackage[T1]{fontenc}
\usepackage{times}
\usepackage{calligra}

\usepackage{subcaption}
\usepackage{graphicx}

\usepackage{latexsym}
\usepackage{amsmath,amssymb}
%\usepackage{subfigure}

\usepackage{enumerate}
\usepackage{tabulary}
\usepackage{url}

\spanishdecimal{.}

\usepackage{float}
\usepackage{colortbl}

\usepackage{latexsym}
\usepackage{xcolor}
\usepackage{fancyhdr}
\pagestyle{fancy}
\usepackage{balance}

\usepackage[colorlinks=true, allcolors=blue]{hyperref}

\title{Actividades de repaso. Unidad 1\\
Números reales}
\author{Olivia Palma Avedaño \\ \href{mailto:
oliviapalma@gmail.com}{oliviapalma@gmail.com}}

\date{}

\begin{document}

\maketitle
\begin{figure}[t!]
\begin{subfigure}{.5\textwidth}
\flushleft  \includegraphics[width=.3\linewidth]{ENPlogo.jpg}
\end{subfigure}
\begin{subfigure}{.5\textwidth}
\flushright  \includegraphics[width=.3\linewidth]{unam_logo.jpg}
\end{subfigure}
\end{figure}

\begin{center}
  \textit{Escuela Nacional Preparatoria No. 5 José Vasconcelos\\
Universidad Nacional Autónoma de México}


	\vspace*{0.1cm}
	
	\begin{center}\rule{0.9\textwidth}{0.1mm} \end{center}

  
\end{center}

\textbf{Indicaciones:} responder usando números racionales y operaciones con números racionales.

\begin{quote}
\textit{Miguel camina diariamente $7$ $\frac{3}{4}$ $km$. Un día de la semana, caminó la mitad de su recorrido porque estaba lloviendo. El viernes llegó su amigo y caminaron $\frac{1}{3}$ más de lo habitualmente caminado por Miguel.}
\end{quote}

    \textbf{Preguntas} (valor: 2 puntos cada una).\\

\begin{enumerate}[(a)]
\item ¿Cuánto caminó el día lluvioso?
\item ¿Cuánto caminó el día que vino su amigo?
\item ¿Cuánto camina los domingos?
\item En total de kilómetros en esa semana que caminó Miguel es de:
\item Si su médico le dice que semanalmente debe de caminar $30 \, km$, ¿se excedió de kilómetros o le faltaron? ¿Cuánto o cuántos?
\end{enumerate}





\end{document}
