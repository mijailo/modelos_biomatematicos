\documentclass[letterpaper,9pt]{article}
\usepackage[margin=1.5cm]{geometry}
\usepackage[spanish]{babel}
\usepackage[utf8]{inputenc}
\usepackage[T1]{fontenc}
\usepackage{times}
\usepackage{calligra}
\usepackage{graphicx}

\usepackage{latexsym}
\usepackage{amsmath,amssymb}
\usepackage{subfigure}

\usepackage{tabulary}
\usepackage{url}

\spanishdecimal{.}

\usepackage{float}
\usepackage{colortbl}

\usepackage{latexsym}
\usepackage{xcolor}
\usepackage{fancyhdr}
\pagestyle{fancy}
\usepackage{balance}

\usepackage[colorlinks=true, allcolors=blue]{hyperref}


\title{Rúbrica para la evaluación de actividad de desarrollo}
\author{M. en C. Manuel Mijaíl Martínez Ramos}

\date{}

\begin{document}
\maketitle

\begin{table}[h!]
    \centering
    \begin{tabular}{||m{2.5cm}||m{3.5cm}|m{3.5cm}|m{3.5cm}|m{3.5cm}||}
    \hline
    \hline
     \textbf{Criterios}  &  \textbf{Muy bueno} & \textbf{Bueno} & \textbf{Regular} & \textbf{Insuficiente} \\
         \textbf{Puntos}  & \textit{4 puntos} & \textit{3 puntos} & \textit{2 puntos} &  \textit{1 punto}\\
         \hline
         \hline
        \textbf{Comprensión del problema} & Analiza, reconoce e interpreta perfectamente los datos, identificando con certeza lo que se quiere resolver y demostrando una muy buena comprensión del problema. & Analiza, reconoce e interpreta perfectamente los datos, identificando con claridad lo que se busca y demuestra una buena comprensión del problema. & Reconoce los datos e identifica la relación entre los mismos. Demuestra una comprensión elemental del problema. & No reconoce los datos, sus relaciones ni el contexto del problema. Por lo que muestra muy poca comprensión del mismo. \\
         \hline
        \textbf{Representación algebraica} & Relaciona de manera eficiente las variables involucradas en el problema y construye modelos matemáticos sencillos con la información dada.  & Relaciona de manera eficiente las variables involucradas en el problema. Muestra cierta dificultad para llegar al modelo matemático. & La relación algebraica entre las variables que muestra, no es correcta. & No encuentra relaciones entre las variables involucradas en el problema. \\
         \hline
        \textbf{Cálculos y operaciones} & Detalla los pasos seguidos relacionando y aplicando los conceptos matemáticos necesarios. & Se salta pasos, aunque sí aplica los conceptos matemáticos necesarios. & Los pasos y operaciones no son claros, no siempre utiliza adecuadamente los conceptos matemáticos. &  Los pasos y operaciones no son correctos. No utiliza los conceptos matemáticos de manera adecuada.\\
         \hline
         \textbf{Resultados aplicados al contexto} & Da de manera correcta la solución del problema y los vincula con el contexto. & Da los resultados de manera correcta, aunque la vinculación con el contexto no es clara. & Algunos resultados no son correctos y no hay vinculación adecuada con el contexto & Los resultados dados no son correctos y no hay vinculación con el contexto. \\
         \hline
         \textbf{Análisis de los resultados} & Analiza y discute sobre la solución, reflexiona y valora sobre su fiabilidad. Revisa el proceso, detecta si hay errores y los rectifica. & Analiza y discute sobre la solución, reflexiona y valora sobre su fiabilidad. & No hace un buen análisis sobre la solución y no hace valoración sobre la fiabilidad. & No hace análisis ni discusión sobre la solución. \\
         \hline
         \textbf{Comunicación de los resultados} & Expresa de manera clara las razones que lo llevaron a tomar esa decisión al resolver el problema. & Al expresar las razones que lo llevaron a tomar esa decisión, necesita ayuda del profesor o de un integrante del equipo. & No logra expresar de manera clara las razones que lo llevaron a resolver el problema aun con apoyo. & No expresa los resultados. \\
         \hline
         \hline
    \end{tabular}
\end{table}

\end{document}
